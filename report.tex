\documentclass[12pt]{article}

\usepackage{fullpage}
\usepackage{multicol,multirow}
\usepackage{tabularx}
\usepackage{ulem}
\usepackage[utf8]{inputenc}
\usepackage[russian]{babel}
\usepackage{minted}

% Оригиналный шаблон: http://k806.ru/dalabs/da-report-template-2012.tex

\begin{document}
\begin{titlepage}
\begin{center}
\textbf{МИНИСТЕРСТВО ОБРАЗОВАНИЯ И НАУКИ РОССИЙСОЙ ФЕДЕРАЦИИ
\medskip
МОСКОВСКИЙ АВЦИАЦИОННЫЙ ИНСТИТУТ
(НАЦИОНАЛЬНЫЙ ИССЛЕДОВАТЬЕЛЬСКИЙ УНИВЕРСТИТЕТ)
\vfill\vfill
{\Huge ЛАБОРАТОРНАЯ РАБОТА №4} 
по курсу объектно-ориентированное программирование
I семестр, 2019/20 уч. год}
\end{center}
\vfill

Студент \uline{\it {Попов Данила Андреевич, группа 08-208Б-18}\hfill}

Преподаватель \uline{\it {Журавлёв Андрей Андреевич}\hfill}

\vfill
\end{titlepage}

\subsection*{Условие}

Задание №1: написать программу, которая реализует работу с фигурами: 
\begin{enumerate}
\item Ввод фигуры
\item Ввод фигуры как tuple
\end{enumerate}

\subsection*{Описание программы}

Код программы состоит из 4-ти файлов:
\begin{enumerate}
\item app/main.cpp: исходный код с точкой входа
\item include/algorithm.hpp: объявление и реализация generic структур
\item include/polygon.hpp: объявление и реализация generic структур
\item inclue/point.hpp: объявление и реализация generic структур
\end{enumerate}

\subsection*{Выводы}
Шаблоны Visual C++ не полностью совместимы с шаблонами из GCC.
\vfill

\subsection*{Исходный код}


%{\Huge main.hpp}
%\inputminted
%{C++}{app/main.cpp}
%\pagebreak

{\Huge algorithm.hpp}
\inputminted
{C++}{include/algorithm.hpp}
\pagebreak

{\Huge polygon.hpp}
\inputminted
{C++}{include/polygon.hpp}
\pagebreak

{\Huge point.hpp}
\inputminted
{C++}{include/point.hpp}
\pagebreak

\end{document}
